\documentclass{article}[UTF8]

\usepackage{polski}
\usepackage[utf8]{inputenc}
\usepackage{times}
\usepackage{graphicx}
\usepackage{tabularx}
\usepackage{latexsym}
\usepackage{amssymb}
\usepackage{amsthm}
\usepackage{amsmath}
\usepackage{multirow}
%\usepackage{theorem}

%%%%%%%%%%%%%%%%%%%%%%%%%%%%%%% MACRO

\renewcommand{\S}{\mathcal{S}}
\newcommand{\pay}{\mathit{pay}}
\newcommand{\N}{\mathcal{N}}
\newcommand{\s}{\mathbf{s}}
\renewcommand{\b}{\mathbf{b}}
\renewcommand{\r}{\mathbf{r}}
\newcommand{\w}{\mathbf{w}}
\renewcommand{\P}{\mathcal{P}}

\newcommand{\G}{\mathcal{G}}
\newcommand{\U}{\mathcal{U}}
\newcommand{\tuple}[1]{\langle#1\rangle}
\newcommand{\todo}[1]{\noindent{{\small\bf$<$\textsf{#1}$>$}}}

\renewcommand{\sp}{\mathsf{sp}}
\newcommand{\mcs}{\mathsf{cs}}
\newcommand{\ssp}{\mathsf{ssp}}
\newcommand{\SC}{\mathsf{SC}}
\newcommand{\NE}{\mathsf{NE}}
\newcommand{\SE}{\mathsf{SE}}

%ALGORITMO
\newcounter{stepcount}
\newcommand{\step}{%
        \stepcounter{stepcount}%
        \thestepcount}
\newcommand{\resetstep}{%
        \setcounter{stepcount}{0}
}

\newtheoremstyle{break}
  {\topsep}{\topsep}%
  {\itshape}{}%
  {\bfseries}{}%
  {\newline}{}%
\theoremstyle{definition}
\newtheorem{zad}{Zadanie}


%\newtheorem{example}{Example}[section]
%\newtheorem{definition}{Definition}[section]
%\newtheorem{lemma}{Lemma}[section]
%\newtheorem{theorem}{Theorem}[section]

%\newenvironment{proof}{\vspace{-1mm}{\em Proof.}}{\vspace{2mm}}

\newcolumntype{Y}{>{\centering\arraybackslash}X}

\begin{document}
\title{Badania operacyjne i systemy wspomagania decyzji
\\ \Large 12-13 AHP: rankingi i niespójność}
\date{}

\maketitle

\begin{zad}

Rodzina Kowalskich postanowiła zakupić dom. Po dłuższych poszukiwaniach znaleźli trzy oferty, które ich zainteresowały. Nie mogąc się zdecydować jednoznacznie na którąkolwiek z nich, postanowili porównać oferty parami w następujących kategoriach:
1 - cena,
2 - rozmiar domu,
3 - dostęp do komunikacji miejskiej,
4 - dzielnica,
5 - wiek domu,
6 - rozmiar ogródka,
7 - wyposażenie,
8 - stan techniczny.
Efekty tych porównań, opisane przy użyciu fundamentalnej skali różnic, są zapisane w poniższych macierzach:


$$C_1=\begin{pmatrix}
1 & \frac{1}{7} & \frac{1}{5} \\
7&1&3 \\
5&\frac{1}{3}&1
\end{pmatrix}, \
C_2=\begin{pmatrix}
1&5&9\\
\frac{1}{5}&1&4\\
\frac{1}{9}&\frac{1}{4}&1
\end{pmatrix}, \
C_3=\begin{pmatrix}
1&4&\frac{1}{5}\\
\frac{1}{4}&1&\frac{1}{9}\\
5&9&1
\end{pmatrix},
$$
$$
C_4=\begin{pmatrix}
1&9&4\\
\frac{1}{9}&1&\frac{1}{4}\\
\frac{1}{4}&4&1
\end{pmatrix}, \
C_5=\begin{pmatrix}
1&1&1\\
1&1&1\\
1&1&1
\end{pmatrix}, \
C_6=\begin{pmatrix}
1&6&4\\
\frac{1}{6}&1&\frac{1}{3}\\
\frac{1}{4}&3&1
\end{pmatrix},
$$
$$C_7=\begin{pmatrix}
1&9&6\\
\frac{1}{9}&1&\frac{1}{3}\\
\frac{1}{6}&3&1
\end{pmatrix}, \
C_8=\begin{pmatrix}
1&\frac{1}{2}&\frac{1}{2}\\
2&1&1\\
2&1&1
\end{pmatrix}.
$$

Porównanie na ile poszczególne kategorie są względem siebie dla nich ważne zostało opisane w poniższej macierzy:

$$C_{parametry}=\begin{pmatrix}
1&4&7&5&8&6&6&2\\
\frac{1}{4}&1&5&3&7&6&6&\frac{1}{3}\\
\frac{1}{7}&\frac{1}{5}&1&\frac{1}{3}&5&3&3&\frac{1}{5}\\
\frac{1}{5}&\frac{1}{3}&3&1&6&3&4&\frac{1}{2}\\
\frac{1}{8}&\frac{1}{7}&\frac{1}{5}&\frac{1}{6}&1&\frac{1}{3}&\frac{1}{4}&\frac{1}{7}\\
\frac{1}{6}&\frac{1}{6}&\frac{1}{3}&\frac{1}{3}&3&1&\frac{1}{2}&\frac{1}{5}\\
\frac{1}{6}&\frac{1}{6}&\frac{1}{3}&\frac{1}{4}&4&2&1&\frac{1}{5}\\
\frac{1}{2}&3&5&2&7&5&5&1
\end{pmatrix}
$$

Należy wyznaczyć rankingi poszczególnych macierzy porównań, ranking kategorii oraz globalny ranking alternatyw, używając metod EVM i GMM.


\end{zad}

\begin{zad}

Policz indeksy spójności Satty'ego oraz indeks niespójności Koczkodaja dla poniższych macierzy:

$$
A = \begin{pmatrix}
1&7&3\\
\frac{1}{7}&1&2\\
\frac{1}{3}&\frac{1}{2}&1
\end{pmatrix},
B=\begin{pmatrix}
1&\frac{1}{5}&7&1\\
5&1&\frac{1}{2}&2\\
\frac{1}{7}&2&1&3\\
1&\frac{1}{2}&\frac{1}{3}&1
\end{pmatrix},
C=\begin{pmatrix}
1&2&5&1&7\\
\frac{1}{2}&1&3&\frac{1}{2}&5\\
\frac{1}{5}&\frac{1}{3}&1&\frac{1}{5}&2\\
1&2&5&1&7\\
\frac{1}{7}&\frac{1}{5}&\frac{1}{2}&\frac{1}{7}&1
\end{pmatrix}
$$


\end{zad}

\end{document}
%%%%%%%%%%%%%%%%%%%%%%%%%%%%%%%%%%%%%%%%%%%%%%%%%%%%%%%%%%%%%%%%%%%%%%
