\documentclass{article}[UTF8]

\usepackage{polski}
\usepackage[utf8]{inputenc}
\usepackage{times}
\usepackage{graphicx}
\usepackage{tabularx}
\usepackage{latexsym}
\usepackage{amssymb}
\usepackage{amsthm}
\usepackage{amsmath}
\usepackage{multirow}
%\usepackage{theorem}

%%%%%%%%%%%%%%%%%%%%%%%%%%%%%%% MACRO

\renewcommand{\S}{\mathcal{S}}
\newcommand{\pay}{\mathit{pay}}
\newcommand{\N}{\mathcal{N}}
\newcommand{\s}{\mathbf{s}}
\renewcommand{\b}{\mathbf{b}}
\renewcommand{\r}{\mathbf{r}}
\newcommand{\w}{\mathbf{w}}
\renewcommand{\P}{\mathcal{P}}

\newcommand{\G}{\mathcal{G}}
\newcommand{\U}{\mathcal{U}}
\newcommand{\tuple}[1]{\langle#1\rangle}
\newcommand{\todo}[1]{\noindent{{\small\bf$<$\textsf{#1}$>$}}}

\renewcommand{\sp}{\mathsf{sp}}
\newcommand{\mcs}{\mathsf{cs}}
\newcommand{\ssp}{\mathsf{ssp}}
\newcommand{\SC}{\mathsf{SC}}
\newcommand{\NE}{\mathsf{NE}}
\newcommand{\SE}{\mathsf{SE}}

%ALGORITMO
\newcounter{stepcount}
\newcommand{\step}{%
        \stepcounter{stepcount}%
        \thestepcount}
\newcommand{\resetstep}{%
        \setcounter{stepcount}{0}
}

\newtheoremstyle{break}
  {\topsep}{\topsep}%
  {\itshape}{}%
  {\bfseries}{}%
  {\newline}{}%
\theoremstyle{definition}
\newtheorem{zad}{Zadanie}


%\newtheorem{example}{Example}[section]
%\newtheorem{definition}{Definition}[section]
%\newtheorem{lemma}{Lemma}[section]
%\newtheorem{theorem}{Theorem}[section]

%\newenvironment{proof}{\vspace{-1mm}{\em Proof.}}{\vspace{2mm}}

\newcolumntype{Y}{>{\centering\arraybackslash}X}

\begin{document}
\title{Badania operacyjne i systemy wspomagania decyzji
\\ \Large 12 AHP - metoda EVM}
\date{}

\maketitle


\begin{zad}

Rodzina Kowalskich postanowiła zakupić dom. Po dłuższych poszukiwaniach znaleźli trzy oferty, które ich zainteresowały. Nie mogąc się zdecydować jednoznacznie na którąkolwiek z nich, postanowili porównać oferty parami w następujących kategoriach:
1 - cena,
2 - rozmiar domu,
3 - dostęp do komunikacji miejskiej,
4 - dzielnica,
5 - wiek domu,
6 - rozmiar ogródka,
7 - wyposażenie,
8 - stan techniczny.
Efekty tych porównań, opisane przy użyciu fundamentalnej skali różnic, są zapisane w poniższych macierzach:


$$C_1=\begin{pmatrix}
1 & \frac{1}{7} & \frac{1}{5} \\
7&1&3 \\
5&\frac{1}{3}&1
\end{pmatrix}, \
C_2=\begin{pmatrix}
1&5&9\\
\frac{1}{5}&1&4\\
\frac{1}{9}&\frac{1}{4}&1
\end{pmatrix}, \
C_3=\begin{pmatrix}
1&4&\frac{1}{5}\\
\frac{1}{4}&1&\frac{1}{9}\\
5&9&1
\end{pmatrix},
$$
$$
C_4=\begin{pmatrix}
1&9&4\\
\frac{1}{9}&1&\frac{1}{4}\\
\frac{1}{4}&4&1
\end{pmatrix}, \
C_5=\begin{pmatrix}
1&1&1\\
1&1&1\\
1&1&1
\end{pmatrix}, \
C_6=\begin{pmatrix}
1&6&4\\
\frac{1}{6}&1&\frac{1}{3}\\
\frac{1}{4}&3&1
\end{pmatrix},
$$
$$C_7=\begin{pmatrix}
1&9&6\\
\frac{1}{9}&1&\frac{1}{3}\\
\frac{1}{6}&3&1
\end{pmatrix}, \
C_8=\begin{pmatrix}
1&\frac{1}{2}&\frac{1}{2}\\
2&1&1\\
2&1&1
\end{pmatrix}.
$$

Porównanie na ile poszczególne kategorie są względem siebie dla nich ważne zostało opisane w poniższej macierzy:

$$C_{parametry}=\begin{pmatrix}
1&4&7&5&8&6&6&2\\
\frac{1}{4}&1&5&3&7&6&6&\frac{1}{3}\\
\frac{1}{7}&\frac{1}{5}&1&\frac{1}{3}&5&3&3&\frac{1}{5}\\
\frac{1}{5}&\frac{1}{3}&3&1&6&3&4&\frac{1}{2}\\
\frac{1}{8}&\frac{1}{7}&\frac{1}{5}&\frac{1}{6}&1&\frac{1}{3}&\frac{1}{4}&\frac{1}{7}\\
\frac{1}{6}&\frac{1}{6}&\frac{1}{3}&\frac{1}{3}&3&1&\frac{1}{2}&\frac{1}{5}\\
\frac{1}{6}&\frac{1}{6}&\frac{1}{3}&\frac{1}{4}&4&2&1&\frac{1}{5}\\
\frac{1}{2}&3&5&2&7&5&5&1
\end{pmatrix}
$$
\end{zad}

Na podstawie powyższych informacji wykorzystaj metodę AHP do stworzenia rankingu domów. 
\newpage

\begin{zad}
Doktor A. wybiera się na konferencję do Wrocławia. Przez organizatorów konferencji proponowane są następujące hotele: \\

\textbf{Hotel TUR:}\\
1.cena: 210 zł/noc\\
2.wyżywienie: 20 zł \\
3.odległość od konferencji: 20 minut pieszo; 3 przystanki\\
4.parking przy hotelu: nie

\textbf{Hotel KUR:}\\
1.cena: 150 zł/noc\\
2.wyżywienie: 20 zł\\
3.odległość od konferencji: 30 minut pieszo; 7 przystanków, 1 przesiadka \\
4.parking przy hotelu: tak

\textbf{Hotel MUR:}\\
1.cena: 230 zł/noc\\
2.wyżywienie:30 zł\\
3.odległość od konferencji: 12 minut pieszo; 2 przystanki\\
4.parking przy hotelu: nie

\textbf{Hotel BÓR:}\\
1.cena: 250 zł/noc\\
2.wyżywienie: 25zł\\
3.odległość od konferencji: 8 minut pieszo; 1 przystanek\\
4.parking przy hotelu: tak

Zapytany o wpływ poszczególnych kategorii na jego potencjalny wybór podał następującą macierz porównań:

$$C=\begin{pmatrix}
1&5&3&4\\
\frac{1}{5}&1&4&1\\
\frac{1}{3}&\frac{1}{4}&1&2\\
\frac{1}{4}&1&\frac{1}{2}&1
\end{pmatrix}.
$$


Na podstawie powyższych informacji zaproponuj macierze porównań kategorii w oparciu o fundamentalną skalę różnic i wykorzystaj metodę AHP do stworzenia rankingu hoteli.

\end{zad}

\newpage

\begin{zad}

W poniższym zadaniu samodzielnie wykonaj wszystkie kroki potrzebne to stworzenia rankingu samochodów metodą AHP korzystając z proponowanego podziału na kryteria i podkryteria.

Samochód 1:\\
Cena samochodu: 68 200zł\\
Zużycie paliwa: 8,9l/100km\\
Bezpieczeństwo: 4/5\\
Rozmiar bagażnika: 460l\\
Ilość pasażerów: 5

Samochód 2:\\
Cena samochodu: 39 900zł\\
Zużycie paliwa: 11,2l/100km\\
Bezpieczeństwo: 3,5/5\\
Rozmiar bagażnika: 415l\\
Ilość pasażerów: 5

Samochód 3:\\
Cena samochodu: 87 394zł\\
Zużycie paliwa: 5,8l/100km\\
Bezpieczeństwo: 4,5/5\\
Rozmiar bagażnika: 430l\\
Ilość pasażerów: 4

Proponowany podział na kryteria i podkryteria dla metody AHP:\\
Kryteria: cena, bezpieczeństwo, pojemność.\\
Podkryteria ceny: cena samochodu, zużycie paliwa.\\
Podkryteria pojemności: rozmiar bagażnika, ilość pasażerów. 

\end{zad}
\end{document}
%%%%%%%%%%%%%%%%%%%%%%%%%%%%%%%%%%%%%%%%%%%%%%%%%%%%%%%%%%%%%%%%%%%%%%
