\documentclass{article}[]

\usepackage{polski}
\usepackage[utf8]{inputenc}
\usepackage{times}
\usepackage{graphicx}
\usepackage{tabularx}
\usepackage{latexsym}
\usepackage{amssymb}
\usepackage{amsthm}
\usepackage{amsmath}
\usepackage{multirow}
%\usepackage{theorem}

%%%%%%%%%%%%%%%%%%%%%%%%%%%%%%% MACRO

\renewcommand{\S}{\mathcal{S}}
\newcommand{\pay}{\mathit{pay}}
\newcommand{\N}{\mathcal{N}}
\newcommand{\s}{\mathbf{s}}
\renewcommand{\b}{\mathbf{b}}
\renewcommand{\r}{\mathbf{r}}
\newcommand{\w}{\mathbf{w}}
\renewcommand{\P}{\mathcal{P}}

\newcommand{\G}{\mathcal{G}}
\newcommand{\U}{\mathcal{U}}
\newcommand{\tuple}[1]{\langle#1\rangle}
\newcommand{\todo}[1]{\noindent{{\small\bf$<$\textsf{#1}$>$}}}

\renewcommand{\sp}{\mathsf{sp}}
\newcommand{\mcs}{\mathsf{cs}}
\newcommand{\ssp}{\mathsf{ssp}}
\newcommand{\SC}{\mathsf{SC}}
\newcommand{\NE}{\mathsf{NE}}
\newcommand{\SE}{\mathsf{SE}}

%ALGORITMO
\newcounter{stepcount}
\newcommand{\step}{%
        \stepcounter{stepcount}%
        \thestepcount}
\newcommand{\resetstep}{%
        \setcounter{stepcount}{0}
}

\newtheoremstyle{break}
  {\topsep}{\topsep}%
  {\itshape}{}%
  {\bfseries}{}%
  {\newline}{}%
\theoremstyle{definition}
\newtheorem{zad}{Zadanie}


%\newtheorem{example}{Example}[section]
%\newtheorem{definition}{Definition}[section]
%\newtheorem{lemma}{Lemma}[section]
%\newtheorem{theorem}{Theorem}[section]

%\newenvironment{proof}{\vspace{-1mm}{\em Proof.}}{\vspace{2mm}}

\newcolumntype{Y}{>{\centering\arraybackslash}X}

\begin{document}
\title{Badania operacyjne i systemy wspomagania decyzji
\\ \Large 02 Programowanie liniowe}
\date{}

\maketitle

\begin{zad}
Dany jest program liniowy
\begin{align*}
	x_1+x_2+x_3 &\leq 30 \\
	x_1+2x_2+x_3 &\geq 10 \\
	2x_2+x_3 &\leq 20 \\
	x_1,x_2,x_3 &\geq 0 \\
	2x_1+x_2+3x_3 &\rightarrow \max.
\end{align*}

Sprowadź ten program do postaci kanonicznej i znajdź rozwiązanie.

\end{zad}

\begin{zad}
	Racjonalna hodowla drobiu wymaga dostarczenia każdej sztuce dwóch składników odżywczych: $S_1$ i $S_2$ w ilościach nie mniejszych niż odpowiednio 1200 i 600 jednostek Składniki te zawarte są w czterech paszach: $P_1$, $P_2$, $P_3$ i $P_4$
	
	\begin{table}[htbp]
		\begin{center}
			
			\begin{tabularx}{\textwidth}{|Y|Y|Y|Y|}
				\hline
				\multirow{2}{*}{Pasze} & \multicolumn{2}{|c|}{Zawartość w 1 kg paszy składnika} &Cena 1 kg  \\
				\cline{2-3}
				& $S_1$ & $S_2$ &paszy w zł\\
				\hline
				$P_1$ & 0,8 & 0,6 & 9,6\\
				$P_2$ & 2,4 & 0,6 & 14,4\\
				$P_3$ & 0,9 & 0,3 & 10,8\\
				$P_4$ & 0,4 & 0,3 & 7,2\\
				\hline
				
			\end{tabularx}
		\end{center}
	\end{table}
	
	W jakich ilościach zakupić poszczególne pasze, aby dostarczyć niezbędne składniki odżywcze przy możliwie najniższych kosztach zakupu pasz?
	
\end{zad}

\begin{zad}

Rudolf Edmund uwielbia steki i ziemniaki, dlatego też postanowił rozpocząć dietę opartą głównie na tych dwóch składnikach (z małym dodatkiem płynów i suplementów). Rudolf zdaje sobie sprawę, że nie jest to najzdrowsza dieta, dlatego postanowił je spożywać w ilościach, które zaspokoją mu przynajmniej pewne podstawowe potrzeby żywieniowe. Napisz program liniowy minimalizujący koszt takiej diety i znajdź rozwiązanie.

\begin{table}[htbp]
\begin{center}

\begin{tabularx}{12cm}{|Y|Y|Y|Y|}
\hline
& \multicolumn{2}{|c|}{Gramy składnika w jednej porcji}& Dzienne \\
\cline{2-3}
Składnik & Steki & Ziemniaki& zapotrzebowanie w gramach\\
\hline
Węglowodany & 5 & 15 & $\geq$ 50\\
Białka & 20 & 5 & $\geq$ 40 \\
Tłuszcze & 15 & 2 & $\leq$ 60 \\
\hline
Cena za porcję & 8 zł & 4 zł &\\
\hline

\end{tabularx}
\end{center}
\end{table}

\end{zad}

\begin{zad}
	Fabryka celulozy i papieru musi wyprodukować dokładnie 150 zwojów papieru o szerokości 105 cm, 200 zwojów papieru o szerokości 75 cm i 150 zwojów o szerokości 35 cm. Jako surowiec zostanie użyty papier zrolowany o szerokości 2 m. W jaki sposób fabryka ma zrealizować zamówienie tak, żeby odpad z procesu cięcia był jak najmniejszy? 
\end{zad}

\end{document}
%%%%%%%%%%%%%%%%%%%%%%%%%%%%%%%%%%%%%%%%%%%%%%%%%%%%%%%%%%%%%%%%%%%%%%
